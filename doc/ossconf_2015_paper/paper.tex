\documentclass[twoside]{oss-conf-eng}
    % for slovak or czech use \documentclass[twoside]{oss-conf}
\usepackage[english,slovak]{babel}
    % use english and your language respectively ... czech, polish, ...
\usepackage[T1]{fontenc}
\usepackage[utf8]{inputenc}
\usepackage{amsmath}
\usepackage{amscd,amssymb,amsfonts}
\usepackage[sort,nocompress]{cite}
\usepackage{graphicx}                          % use for include pictures *.jpg, *.png, *.pdf
  %\usepackage{color}
  %\usepackage{colortbl}
  %\usepackage{multicol}
  %\usepackage{comment}
\usepackage{fancyhdr}
\usepackage[pdftex,unicode,bookmarks=false]{hyperref}  % for e-mail and url adress
\usepackage{listings}
\usepackage{eurosym}
\usepackage{lmodern}


%.............................................................................
\begin{document}
\def\keywordsnameS{Keywords}
%kohonen neural network, learning controll, iterative learning, PID controll, mobile robot}
%\def\keywordsnameS{Klíčová slova}    % pre češtinu odpoznámkovať

\setcounter{page}{1}                            % number of first page of article
\def\volumeDOI{OSSConf 2014:\  }                % proceedings information
\def\konfera{Konferencia OSSConf 2014}


\pagestyle{fancy}
\fancyfoot{}
\fancyhead[LE,RO]{\thepage}
\fancyhead[LO]{\nouppercase{\leftmark: \rightmark}}
\fancyhead[RE]{\nouppercase{\konfera}}


%.............................................. slovak or czech title
%          for shorts titles use command   \title{CELÝ NÁZOV ČLÁNKU}
\title[Klasifikácia žiadanej hodnoty Kohonenovou neurónovou sieťou]{Klasifikácia žiadanej hodnoty Kohonenovou neurónovou sieťou}


%.............................................. english title
%          for shorts titles use command   \titleA{THE TITLE OF ARTICLE}
\titleA[Klasifikácia žiadanej hodnoty Kohonenovou neurónovou sieťou]{Required value classification using Kohonen neural network}


%............................. authors (first, second, ...) .. use next commands:
%
% \author[shorts]{Name Surname}{Title}{Past Title}{Country code}
% \address{address}
% \curraddress{current address}    alternative address
% \emailh{sanonymous@water.sun.xy}
% \urladdressh{http://www.sanous.tows.xy/~sanonymous}


\author[M. Chovanec]{Michal Chovanec}{Ing.}{}{}
\address{Department of Technical Cybernetics, Faculty of Management Science and Informatics,
University of Žilina, Univerzitná~8215/1, 010~26~Žilina, Slovak Republic}
\curraddress{SOIT, 010~01 Žilina, Slovak Republic}
\emailh{michal.chovanec@yandex.com}
\urladdressh{http://aikenshin.blogspot.sk}

%.............................................. dedicatory
% \dedicatory{Dedicated to my hound\dots, optional}


%.............................................. keywords slovak/english
\keywords{kohonenová neurónová sieť, inteligentné učenie, iteraivne učenie, PID riadenie, mobilný robot}
\keywordsA{kohonen neural network, learning controll, iterative learning, PID controll, mobile robot}

%.............................................. (obligatory) AMS Classification 2000
% The 1-st classification is obligatory, the 2-nd classification is optional
% \subjclass{primary}{secondary}      f.e. \subjclass{35R35, 49N50}{}
  % \subjclass{35R35, 49M15, 49N50}{}



%.............................................. slovak Abstract
\selectlanguage{slovak}

\begin{abstract}
V článku je popísaná klasifikácia situacií Kohonenovou neurónovou sieťou.
Výsledky sú demonštrované na robotovi, sledujúceho čiaru, kde sú klasifikované rôzne
typi zákrut. Každá situácia má zodpovedajúci výstup, ktorý je vstupom do nižšej úrovne riadenia.
\end{abstract}


%.............................................. english Abstract
\selectlanguage{english}

\begin{abstractA}
In this paper we describe situations classification using Kohonen neural network.
Results are demonstrated on line following robot, where different curves are classificated.
Each classificated situation have corresponding required value output,
which go into low level controll process.
Network weights estimating is processed in real time.
\end{abstractA}



\selectlanguage{english}
%........................... private macros (f.e. the different environments)
\newtheorem{theorem}{Theorem}[section]
\newtheorem{corollary}[theorem]{Corollary}
\newtheorem{lemma}[theorem]{Lemma}
\newtheorem{exmple}[theorem]{Example}
\newtheorem{defn}[theorem]{Definition}
\newtheorem{proposition}[theorem]{Proposition}
\newtheorem{conjecture}[theorem]{Conjecture}
\newtheorem{rmrk}[theorem]{Remark}
            %%% for no-italic, numbered environments, use:
\newenvironment{definition}{\begin{defn}\normalfont}{\end{defn}}
\newenvironment{remark}{\begin{rmrk}\normalfont}{\end{rmrk}}
\newenvironment{example}{\begin{exmple}\normalfont}{\end{exmple}}
            %%% for unnumbered environments, use f.e.
\newtheorem*{remarque}{Remark}
\newcommand\bbreak{\allowdisplaybreaks}
\newcommand\dd{\mathop{\rm d\!}\nolimits}
\newcommand\sgn{\mathop{\rm sgn}\nolimits}


%.............................................. maketitle
\maketitle

%.............................................. the main article

\section*{Introduction}

Is well know, iterative learning controll can handle any repetitive controll
process weel \cite{ilc1}, \cite{ilc2}. Most of aplications we can found in manipulators
controll in industry.

Some problems, have iterative (or semiterative) character, but more then one required
output sequence is required. Consider problem of line following robot : to controll robot
speed, is necessary to know curve shape. If we can compute asociative memory, where input is
line shape and output is required speed, we can handle this problem.

\section{Problem formalisation}

Consider two wheel (two motors) robot with diferential drive \cite{robot_model_1}. For motors
controll signals we can write

\begin{eqnarray}
\begin{split}
\label{robot_equation}
	r(n) &=	v(n) - d(n) \\
    l(n) &=	v(n) + d(n)
\end{split}
\end{eqnarray}

Where \\
$r(n)$ is right motor controll output \\
$l(n)$ is left motor controll output \\
$v(n)$ is common controll input \\
$d(n)$ is difference controll input \\
\\
$d(n)$ can be computed well using PD controller \cite{servo_pd}. Our goal
is to properly estimate $d(n)$ which is speed of robot.
Robot motion equation can be written as
\begin{eqnarray}
\begin{split}
\label{robot_equation}
	\theta(n) &= (1 + \alpha) \theta(n-1) - \alpha \theta(n-2) + b_0 (r(n) - l(n))  \\
    \nu(n) &=  \alpha \nu(n-1) + (1 - \alpha) b_1 (r(n) + l(n))
\end{split}
\end{eqnarray}

Where \\
$\theta$ is robot orientation (Yaw angle) \\
$\nu$ is robot speed \\
$\alpha$ is robot inertia constant, within $(0, 1)$ \\
$b_0$ and $b_1$ are constants depending on wheel distances, wheel diameters, gear ratio, and
maximum speed

If required $\theta$ and $\nu$ are know, we can very well use pure PID controll defining error
as $e_{\theta}(n) = \theta_r(n) - \theta(n)$, $e_{\nu}(n) = \nu_r(n) - \nu(n)$,
In our line following problem, is $\nu_r(n)$ uknow, value $\theta_r(n)$ $\theta_r(n)$ readed from
line position sensor. In fact, inertia of motors is much
smaller than inertia of whole robot, for controll $\theta(n)$ we can still use PD
controller, and required value let be $\theta_r(n) = 0$, which mean robot is straight on line.

Our goal is to properly estimate $v(n)$ and controll robot speed, to give enough time
for PD controll to setup $\theta(n)$. We are minimizing $e_{\theta}(n)$ and maximizing $\nu(n)$
(corespoding to $v(n)$).


%%%%%%%%%%%%%%%%%%%%%%%%%%%%%%%%%%%%%%%%%%%%%%%%%%%%%%%%%%%%%%%%%%%%%%%%%%%%%%
\section{Controller design}

On figure \ref{fig:robot_controller} is block diagram of proposed controller.

\begin{figure}[]
    \centering
    \includegraphics[width=0.8\textwidth]{block_diagram/robot_block.eps}
    \caption{Robot controller}
    \label{fig:robot_controller}
\end{figure}

Input is line poistion from sensor $s(n)$. All values are normalised into
$\langle -1, 1 \rangle$.
Required value of $\theta(n)$ is marked as
$r(n)$. In simpplificated case, can be set to 0. Computed error is input into two
PD controllers (PID in general, I-term is equal to zero, because \ref{robot_equation}
($\theta$ part) have pole on [1, 0]. If I-term is nonzero, there will be double pole on [1, 0] and
system will be unstable).

PD ontroller outputs are mixed using fuzzy mux, which works as

\begin{eqnarray}
\begin{split}
\label{fuzzy_mux}
	d(n) &= s(n)a(n) + (1 - s(n))b(n)
\end{split}
\end{eqnarray}

Where \\
$a(n)$ is PID A output \\
$b(n)$ is PID B output \\
$s(n)$ is select input, from interval $\langle 0, 1 \rangle$ \\

Select output is produced by nonlinear low pass filtered signal of $e(n)$ as

\begin{eqnarray}
\begin{split}
\label{low_pass}
s(n) =
  \begin{cases}
    s(n-1)k + (1 - k)|e(n)| & \text{if} |e(n)| < s(n-1) \\
    |e(n)| & else
  \end{cases}
\end{split}
\end{eqnarray}

Where $k$ is filter constant, from $(0, 1)$. This filter smooth error
values, and rise to huge output value if error immediately rise.

This system properly switching between two controllers, one for straight line
one for curved line.

To estimate $v(n)$ is Kohonen neural network used \cite{kohonen_neural_network}
 and iterative learning controll. Input into neural networ is vector of $s(n)$ values
(implemented as fifo, in experiments with size M = 16). Mark this vector as $S(n) =
[s(n), ..., s(n - M)]$.

Our goal, is to train network to clasification vectors $S(n)$. Depending on classes count,
we choose coresponding neurons count $N$. In experiments chooses as 16. Which means, 16 different
curves shapes can be recognized.

J-th neuron transfer function can be written as

\begin{eqnarray}
\begin{split}
\label{neuron_transfer}
y_j(n) = \sum_{i = 0}^{M-1} {|s(n-i) - w_j(i)|}
\end{split}
\end{eqnarray}

Where $w_j$ are neuron weights, in initialization chosen randomly
 and modifieng during learning process.

Neuron with less $y_q(n)$ will be marked as winning neuron, and weights modification
will be processed as

\begin{eqnarray}
\begin{split}
\label{neuron_transfer}
w_q(n) = \eta w_q(n-1) + (1 - \eta)S(n)
\end{split}
\end{eqnarray}

Where $\eta$ is learning rate, from $(0, 1)$ close to 1. Weights of neuron are slowly adapted
to corresponding pattern. This can be illustrated on figure \ref{fig:kohonen_test_01}. Input was
2 dimensional, and goal is to find centers (green) of some data set (red). input
data set was produced by some Markov process. On figure \ref{fig:kohonen_test_02} are
show $y_q(n)$ values.

\begin{figure}[]
    \centering
    \includegraphics[width=0.6\textwidth]{kohonen_test/learing_result.png}
    \caption{Kohonen test}
    \label{fig:kohonen_test_01}
\end{figure}

\begin{figure}[]
    \centering
    \includegraphics[width=0.7\textwidth]{kohonen_test/distances_result.png}
    \caption{Kohonen test distances}
    \label{fig:kohonen_test_02}
\end{figure}

After found winning neuron, we need to find corresponding $v(n)$, mark this as
$v_q(n)$.


\begin{eqnarray}
\begin{split}
\label{speed_estimating}
v_q(n) = v_q(n-1) + 1 - k s_f(n)
\end{split}
\end{eqnarray}

This means robot speed $v(n)$ on curve type $q$ is rising if $s_f(n)$ is low.
Where $s_f(n)$ is low pass filtered $s(n)$ value.

\section{Experimental results}

Robot was learning few loops on short line race with different curves types.
Resulting curves types are on figure \ref{fig:curves}. Each line represents
one curve shape. To reduce space, absolute value of $|s(n)|$ as input into
Kohonen neural network has been used. We can seen many straight line positions, and
few curves.

\begin{figure}[]
    \centering
    \includegraphics[width=0.6\textwidth]{prediction/predictor.png}
    \caption{Resulting curves types}
    \label{fig:curves}
\end{figure}

\section{Conclusions}

In paper was describen learning system for curve shape recognition using Kohonen
neural network. Neural network result have coresponding speed output which controll
robot optional speed. Learning process is working in real time,
on 75MHz ARM Cortex M4F microcontroller. Robot is on figure \ref{fig:robot}.
Robot working video can be shown on \cite{robot_video}. More pictures are
availible on authors blog and sources on github \cite{blog_git}.

\begin{figure}[]
    \centering
    \includegraphics[width=0.5\textwidth]{motoko_aftermath_front.jpg}
    \caption{Testing robot}
    \label{fig:robot}
\end{figure}


%============================================================================%
%                                references                                  %
%============================================================================%
\begin{thebibliography}{99}

\bibitem{ilc1}
Kevin L. Moore: Iterative learning controll
\url{http://inside.mines.edu/~kmoore/survey.pdf}.

\bibitem{ilc2}
Kevin L. Moore: An Introduction to Iterative Learning Control
\url{http://inside.mines.edu/~kmoore/ilc-intro-pdftex03-03.pdf}.

\bibitem{robot_model_1}
Patrícia N. Guerra: Linear modelling and idendification of a mobile robot with
differential drive
\url{http://www.dca.ufrn.br/~adelardo/artigos/ICINCO04b.pdf}.

\bibitem{kohonen_neural_network}
R. Rojas: Neural Networks, Springer-Verlag, Berlin, 1996, Kohonen Neural Networks
\url{http://page.mi.fu-berlin.de/rojas/neural/chapter/K15.pdf}

\bibitem{servo_pd}
Michael Goldfarb, Taweedej Sirithanapipat:
The effect of actuator saturation on the performance of PD-controlled servo systems
\url{http://research.vuse.vanderbilt.edu/cim/pubs/journal/35%20-%20Goldfarb%20and%20Sirithanapipat.pdf}.

\bibitem{robot_video}
Michal Chovanec: robot video link
\url{https://www.youtube.com/watch?v=SGaNlbkyQMg}.

\bibitem{blog_git}
Michal Chovanec: authors blog and github
\url{http://aikenshin.blogspot.sk/}.
\url{https://github.com/michalnand/motoko_after_math_linefollower}.

\end{thebibliography}

\end{document}
%%%%%%%%%%%%%%%%%%%%%%%%%%%%%%%%%%%%%%%%%%%%%%%%%%%%%%%%%%%%%%%%%%


%
